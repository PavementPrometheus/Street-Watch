\documentclass[letterpaper,10pt]{article}
\usepackage[utf8]{inputenc}

\title{%
  Problem Statement \\
  \large Pedestrian Counting and Privacy Preservation}
\author{Stephanie Hughes}
\date{CS 461 Fall Term 2018}

\usepackage{natbib}
\usepackage{graphicx}

\usepackage{geometry}
\geometry{textheight=8.5in, textwidth=6in}

\begin{document}
\begin{titlepage}

\maketitle 

\textbf{\large Abstract} \\
The problem my group is working to improve is pedestrian safety in downtown Portland, specifically looking at pedestrian traffic that endangers others through walking in undesignated areas. Through the Traffic Sensor Safety Project, sensors have been installed around several of the deadliest streets in Portland that our group will use to collect data for a machine learning program that can detect pedestrian motion. The project will need to be efficient enough to be used with streetlights and use video data without saving it, along with keeping the identities of the pedestrians private. The project will be measured through the percent accuracy of the pedestrian tracking, the efficiency of the algorithm, and the level of privacy of people’s identities. In the end, the goal of the project is to have tangible suggestions for traffic changes for the city of Portland that can be used to keep pedestrians safe.

\end{titlepage}


\section{Problem Definition}
The problem my group is trying to solve is pedestrian traffic that crosses into vehicle traffic where they are not directed to and endanger people’s safety in downtown Portland. This problem stems from the city of Portland’s desire to have a smart way to track traffic patterns to make pedestrian traffic safer, announced through the Traffic Sensor Safety Project, an article by ZDNet states (S. Condon). Through the Traffic Sensor Safety project, Portland’s most dangerous streets had 200 sensors put up on street lights on SE Division St., SE Hawthorne Blvd. and 122nd St. (S. Condon). These sensors can track the traffic that passes by in order to eliminate congestion and determine best route planning. The city of Portland works to best plan out the traffic ways in order to allow for the most fluidity and safety for all traffic. 

When pedestrians walk in areas where they are not supposed to, they endanger themselves and others, so Portland city planners aim to put walkways where people tend to cross. The city of Portland has installed cameras so that traffic movements could be tracked but there are also privacy laws that must be adhered to. The video itself cannot be saved to the cloud so the video will not be accessible after it is taken by the camera. This creates an obstacle to using the camera as a resource as it does not allow a lot of time to run a program to get the movement data. There needs to be a way to keep the people’s identities private, as the pedestrians should not be able to be identified and tracked if they leave the camera’s view and return; only the movement data should be used. Also, trying to quickly process the camera video instantly could be difficult as the camera is mounted to a street light which does not have that much processing power, making the programs created need to have a certain needed efficiency. 

\section{Proposed Solution}
The proposed solution for this project is to create a program that can use the cameras in place to get pedestrian motion data to determine how traffic ways can be altered so that both pedestrian and vehicle traffic is safe. The program will use machine learning to do image tracking, finding all of the pedestrians in the view of the camera. The camera video that is collected will need to have just the pedestrian movement data taken from it while keeping the identities of the people unknown. This means that the program will need to somehow mask or blur the faces of the people in the video. The program will have to encrypt the video data in such a way that the identification protection is not able to be undone by others. With this, the program should not be able to track pedestrians if they leave the camera frame and come back. The level of privacy of the camera footage is still being determined by the general public, so the project will also consist of gaining human feedback as to how they feel about what they find acceptable. The danger of this kind of project is that people will be tracked or lose their public privacy, so it will be important to create the software with the public opinion in mind. By tracking the movements of the pedestrians, city of Portland planners will be able to find areas where there could be a better and safer design to the traffic flow, and ultimately save people’s lives.

\section{Performance Metrics}
The project will be evaluated in multiple stages with major requirements focused on accuracy of tracking and privacy coverage of pedestrians, in the hopes that our group can deliver tangible traffic efficiency suggestions to the city of Portland. The first area of the project that will be actively measured is the accuracy of the program’s ability to recognize and track pedestrians, with an end goal of achieving at least 70 percent accuracy. Ensuring a high accuracy in pedestrian tracking is important as that data can affect the level of safety that can be achieved through new changes in pedestrian traffic ways. Another important aspect of this project that will be measured is the level of privacy coverage people feel the program has. This measurement will first be recorded through gaining people’s responses on which level of privacy people are most comfortable with when it comes to the video data used. Based on the responses of the people surveyed, our group will develop the software to maintain that level of privacy, with each level of privacy having a certain amount of coverage of one’s identity. People will be surveyed throughout the build cycle so that my group is designing the software with the customer’s needs always in mind. Consistent feedback and evaluation in the level of privacy should by the time we have a final working prototype of the program will leave pedestrians fully satisfied in the level of privacy enabled. The program should also be efficient enough to be able to be used with the amount of processing power the street light provides. Through the accuracy checks, privacy checks, and efficiency checks, our group hopes to have a deliverable solution to provide suggestions for areas of traffic that need to be altered to optimize traffic safety.

\section{Bibliography}
S. Condon, “Portland kicks off smart city initiative with traffic sensor safety project,” ZDNet, 20-Jun-2018. [Online]. Available: https://www.zdnet.com/article/portland-kicks-off-smart-city-initiative-with-traffic-sensor-safety-project/. [Accessed: 12-Oct-2018].

\end{document}
