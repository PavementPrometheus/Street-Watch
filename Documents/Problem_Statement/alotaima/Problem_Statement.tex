\documentclass[onecolumn, draftclsnofoot,10pt, compsoc]{IEEEtran}
\usepackage{graphicx}
\usepackage{url}
\usepackage{setspace}

\usepackage{geometry}
\geometry{textheight=9.5in, textwidth=7in}

% 1. Fill in these details
\def \CapstoneTeamName{		Pavement Prometheus}
\def \CapstoneTeamNumber{		9}
\def \GroupMemberOne{			Mazen Alotaibi}
\def \CapstoneProjectName{		Pedestrian Counting and Privacy Preservation}
\def \CapstoneSponsorCompany{	Oregon State University}
\def \CapstoneSponsorPerson{		Dr. Fuxin Li}

% 2. Uncomment the appropriate line below so that the document type works
\def \DocType{		Problem Statement
				%Requirements Document
				%Technology Review
				%Design Document
				%Progress Report
				}
			
\newcommand{\NameSigPair}[1]{\par
\makebox[2.75in][r]{#1} \hfil 	\makebox[3.25in]{\makebox[2.25in]{\hrulefill} \hfill		\makebox[.75in]{\hrulefill}}
\par\vspace{-12pt} \textit{\tiny\noindent
\makebox[2.75in]{} \hfil		\makebox[3.25in]{\makebox[2.25in][r]{Signature} \hfill	\makebox[.75in][r]{Date}}}}
% 3. If the document is not to be signed, uncomment the RENEWcommand below
\renewcommand{\NameSigPair}[1]{#1}

%%%%%%%%%%%%%%%%%%%%%%%%%%%%%%%%%%%%%%%
\begin{document}
\begin{titlepage}
    \pagenumbering{gobble}
    \begin{singlespace}
    	\includegraphics[height=4cm]{images/coe_v_spot1}
        \hfill 
        % 4. If you have a logo, use this includegraphics command to put it on the coversheet.
        %\includegraphics[height=4cm]{CompanyLogo}   
        \par\vspace{.2in}
        \centering
        \scshape{
            \huge CS Capstone \DocType \par
            {\large\today}\par
            \vspace{.5in}
            \textbf{\Huge\CapstoneProjectName}\par
            \vfill
            {\large Prepared for}\par
            \Huge \CapstoneSponsorCompany\par
            \vspace{5pt}
            {\Large\NameSigPair{\CapstoneSponsorPerson}\par}
            {\large Prepared by }\par
            Group\CapstoneTeamNumber\par
            % 5. comment out the line below this one if you do not wish to name your team
            \CapstoneTeamName\par 
            \vspace{5pt}
            {\Large
                \NameSigPair{\GroupMemberOne}\par
            }
            \vspace{20pt}
        }
        \begin{abstract}
        % 6. Fill in your abstract    
        	Companies in Portland started to invest time and money on building tracking systems to monitor pedestrians, bike riders, and car traffic to understand the behavior of these parameters, however, these systems are limited due technology developments. Therefore,  a company, Smart City PDX, contacted Dr. Fuxin Li to replace the limited tracking systems with a Computer Vision system using surveillance cameras, however, the general public won't feel comfortable to be monitored and identify. Thus, Dr. Li \cite{li} has assigned us to build a Computer Vision system that doesn't disclose the privacy of pedestrian while generating useful traffic data to be used for further analysis.
        \end{abstract}     
    \end{singlespace}
\end{titlepage}
\newpage
\pagenumbering{arabic}
\tableofcontents
% 7. uncomment this (if applicable). Consider adding a page break.
%\listoffigures
%\listoftables
\clearpage

% 8. now you write!
\section{Problem}
% TODO:
% Definition and description of the problem you are trying to solve written for a general, but educated audience
According to Stephanie Condon \cite{zd}, Smart City PDX cameras rely on GE's Current CityIQ sensors, which are powered with Intel IoT technology and use AT\&T as a data carrier. GE, Intel, and AT\&T have already worked together to deploy smart streetlight sensors in San Diego. Smart City PDX is planning to develop a solution by using surveillance cameras and sensors to collect information about the pedestrians and cars' behavior to eliminate fatalities and serious injuries of both pedestrians and cars' passengers. However, Smart City PDX surveillance systems won't work well if a pedestrian doesn't carry a device, a laptop or a phone. Therefore, Hector Dominguez, Open Data Coordinator at Smart City PDX, has communicated with Dr. Fuxin Li, who is our client, to produce a Computer Vision system that tracks pedestrians, bike riders, and car traffic without relying on the availability of cellular devices for pedestrian tracking.

According to Dr. Fuxin Li \cite{li}, the problem is mainly focused on pedestrians, bike riders, and car traffic. Dr. Li \cite{li} suggesting to improve pedestrians, bike riders, and car traffic by making the traffic more safer with faster transportation time by using surveillance cameras to track pedestrians and cars' behavior. The final suggested outcomes by Dr. Li \cite{li} from the project were set up better crosswalks that optimize pedestrians walking style, bikes lanes, and cars lanes. However, Dr. Li \cite{li} suggested that tracking pedestrians using surveillance cameras, the people responsible for monitoring the cameras will be able to identify pedestrians' identity, which is a privacy issue. The general public won't accept to be monitored publicly. There are approximately 200 surveillance cameras on one street in Portland and it is monitored by Smart City PDX.

Although the privacy issue from tracking pedestrians using surveillance cameras is lawful from the government standpoint, the general public doesn't accept the idea for many reasons. First, the general public doesn't want to be monitored and tracked in public area all the time. Second,  although the organization that uses the data is using it ethically, hackers might exploit the system and get valuable information about pedestrians' identity and daily activities.

Dr. Li \cite{li} is interested in increasing the speed of public transportation while keeping traffic safe as public transportation are slow on average due to many factors. For example, the public bus transportation's speed is 14 miles per hour in medium traffic and 19 miles per hour in light traffic, which isn't optimal for the general public. The reason why this public bus transportation isn't optimal isn't that they don't have the mechanical requirements as they can reach the speed of 60 miles per hour at any time, but it is because of the movement and safety of the traffic. The movement and safety of the traffic can be increased by making synchronized traffic lights, correctly located stop signs, optimal bus stop placements, and people should pay remotely rather than cash.



Moreover, 
\section{Proposed Solution}
% TODO:

According to Dr. Li \cite{li}, we will need to find a middle point that will allow us to keep the privacy of pedestrian while generating useful traffic data. We can reach to this middle point by building better Computer Vision systems to mask pedestrians and blur their faces, but the masked pedestrians will need to give us useful information about the behavior of pedestrians and the traffic. Then, we will get the general public opinion about whether the blurred faces are suitable for data gathering and it won't attack their privacy. Finally, we will analyze pedestrians' behavior by using modern Machine Learning models. In addition, when blurring targeted faces we won't be able to deblur these faces, so when hackers compromise the Computer Vision system, they won't be able to identify individuals in each frame.

I have discussed with Dr. Li \cite{li} about our first approach to build the Computer Vision system, which has 4 layers of developments. First, we will use You Only Look Once (YOLOv3) \cite{YOLOv3} for real-time detection, however, YOLOv3 \cite{YOLOv3} doesn't localize objects in videos, which means YOLOv3 \cite{YOLOv3} will only detect targeted objects, but YOLOv3 \cite{YOLOv3} can't differentiate between the detected objects. Therefore, we will use YOLOv3 \cite{YOLOv3} to detect pedestrians in general, then we will mask detected pedestrians for the next step. Second, we will use  Convolutional Neural Networks for Face Recognition (CNN) \cite{CNN} for face recognition and apply different blurring methods. Then, we will use survey methods to extract general public opinion about the blurred faces. Lastly, we will try to use the blurred images to extract pedestrian, bike riders, and car traffic behaviors.

\section{Performance Metrics}
% TODO:
% Performance metrics: Tell how you will know when you have completed the project. Metrics help you and your client agree on what successful completion (e.g., % faster, $-amount cheaper, easier to use, "a working prototype," a complete white paper with research results) of the project looks like.
According to Dr. Li \cite{li}, we have four stages of completion, Pedestrian Detection, Face Detection and Make Masks of Faces, Generate Blurred Faces and Get Public Opinion of the Blurred Faces, and Analyzing Pedestrians' Behavior. Dr. Li \cite{li} suggests we can stop at any stage and we can move to the following stage if we had more time. In addition, Dr. Li \cite{li} proposed that this project is an exploratory project, however, he would prefer to have us built a Computer Vision system that does all of these stages or having a lot of observations and tangible suggestions for what should be implemented to improve current Computer Vision systems.

\bibliographystyle{IEEEtran}  
\bibliography{sources}  

\end{document}