\documentclass[letterpaper,10pt,draftclsnofoot,onecolumn]{IEEEtran}
\usepackage{geometry}
\geometry{margin=0.75in}
\usepackage[utf8]{inputenc}
\fontfamily{cmss}
\usepackage{listings}
\usepackage{xcolor}
\usepackage{titling}
\pretitle{\begin{center}\Huge\bfseries}
\posttitle{\par\end{center}\vskip 0.5em}
\usepackage{setspace} \singlespacing

\title{Privacy and Pedestrian Problems in Portland}
\author{Miles Davies}
\date{}

\begin{document}
\begin{titlingpage}
\maketitle

\thispagestyle{empty}
\begin{center}
CS 461\\
Fall Quarter 2018
\end{center}
\vfill
\begin{abstract}
Managing traffic, vehicular and pedestrian alike, is a problem that afflicts every city in the world. Efforts at approaching this issue are mired in primitive techniques, from relying on volunteers to hand count modes of transportation to shirking any attempt at better transit planning at all. While relying on cameras and archived video footage presents a tantalizing alternative, you’re still left with the conundrum of how best to process said footage and how best to respect civilian privacy. Our project then focuses on how to collect data in a timely manner while simultaneously obfuscating any identifying features of pedestrians and vehicles involved. This information in turn can be used for city planning projects to ensure pedestrian safety or even a smart light system which has been trained on known traffic patterns.
\end{abstract}
\end{titlingpage}

\section{Problem Description}
Traffic is an omnipresent problem in any city, nearly everyone who has ever visited one has at least one experience to share of being waylaid for hours on end, of missing connections, or of being caught in accidents caused by traffic congestion or dangerous road layouts. Clever city planning presents a possible solution, but without adequate information planning safe transit routes becomes an exercise in futility; it also isn’t a full proof solution either, as traffic presents a dynamic problem which ebbs and flows with the city’s resident population. Any given road layout is prone to inefficiencies and dangers given the right conditions- and within a city these conditions change all the time.\par
We require both dynamic solutions and informed transit planning then. While informed transit planning is relatively straightforward it still presents the problem of data collection, likewise a potential dynamic solution for handling traffic includes a smart light and traffic controller system which also requires an adequate data set to be trained and presented with real time information. Using unadulterated video footage for either planning a better transit system or as a training dataset is no silver bullet either, as it can be considered invasive and Big Brotherly by the populace. Processing footage in real time without storage is also problematic, partially because training machine learning algorithms takes considerable time, but also because our most advanced detection technologies only have a detection accuracy of 60-70\%[1]. Not to mention that real time footage is nigh useless for the purposes of designing better road layouts, you might as well have the city planners standing on the city streets.
\section{Proposed Solution}
The Smart City PDX initiative, started by Portland Officials earlier this year, hopes to tackle this traffic congestion and safety problem with the combined efforts of GE, Intel, and AT\&T[2]. Their “Traffic Sensor Safety Project” (which as of yet includes 200 sensors that have been installed on some of Portland’s worst streets) aims to track pedestrians, bicyclists, and cars in order to both understand traffic patterns and eliminate fatalities and injuries associated with traffic congestion while simultaneously respecting the privacy of all parties involved. While they have already had limited success with the smart streetlight sensors of San Diego, Portland’s goal of zero fatalities or serious injuries while maintaining pedestrian privacy remains out of reach[3].\par
Doctor Fuxin Li, of Oregon State University, believes this can be accomplished by using image processing algorithms that can recognize pedestrians and vehicles in real time and then masking them for archived video footage. Masking techniques in turn can be queried against the general public for consideration regarding what might be deemed adequate in maintaining their privacy, which would give us access to video footage which respects the rights of the taped parties involved. This information would have a bevy of applications, from being used to train machine learning algorithms to design a truly responsive traffic controller system that responds in real time to sensor input from the cameras- to helping design a safer transit system with the backing of hard data and statistics.
\section{Performance Metrics}
Our performance metrics are straightforward then, as we want both fast and accurate detection to obfuscate personal, identifying, information so that video footage can be used for city planning projects in order to lower fatalities and injuries associated with traffic. Initial goals would then likely include increasing the accuracy of our best real-time detection system, we would also want to investigate masking techniques that preserve the privacy of civilians. From simple blurring to perhaps representing moving bodies with simple shapes traversing a static image of the street the sensor is recording. We will also need to make surveys and interview pedestrians in order to determine what masking system they’d be most comfortable with being adopted- or at least which masks they consider adequate for purposes of privacy.\par
By the end of this project then we will hopefully have a tool which can be utilized by Portland’s Smart City PDX initiative that will allow real city planning for safer streets to begin. While investigating a smart traffic controller system predicated on machine learning algorithms trained on the collected video footage is tempting, it raises both safety and legal concerns regarding how trustworthy such a system would be- and thereby is likely far outside the scope of this project.
\section{Citations}
\noindent
[1] Hui, J. (2018). {\it Object detection: speed and accuracy comparison}. [online] Medium. Available at: https://medium.com/@jo- nathan\_hui/object-detection-speed-and-accuracy-comparison-faster-r-cnn-r-fcn-ssd-and-yolo-5425656ae359 [Accessed 10 Oct. 2018].\\
\noindent
[2] Smart City PDX. (2018). {\it Smart City Portland}. [online] Available at: https://www.smartcitypdx.com [Accessed 10 Oct. 2018].\\
\noindent
[3] Condon, S. (2018). {\it Portland kicks off smart city initiative with traffic sensor safety project}. [online] ZDNet. Available at: https://www.zdnet.com/article/portland-kicks-off-smart-city-initiative-with-traffic-sensor-safety-project/ [Accessed 10 Oct. 2018].
\end{document}